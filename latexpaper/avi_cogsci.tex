% 
% Annual Cognitive Science Conference
% Sample LaTeX Paper -- Proceedings Format
% 

% Original : Ashwin Ram (ashwin@cc.gatech.edu)       04/01/1994
% Modified : Johanna Moore (jmoore@cs.pitt.edu)      03/17/1995
% Modified : David Noelle (noelle@ucsd.edu)          03/15/1996
% Modified : Pat Langley (langley@cs.stanford.edu)   01/26/1997
% Latex2e corrections by Ramin Charles Nakisa        01/28/1997 
% Modified : Tina Eliassi-Rad (eliassi@cs.wisc.edu)  01/31/1998
% Modified : Trisha Yannuzzi (trisha@ircs.upenn.edu) 12/28/1999 (in process)
% Modified : Mary Ellen Foster (M.E.Foster@ed.ac.uk) 12/11/2000
% Modified : Ken Forbus                              01/23/2004
% Modified : Eli M. Silk (esilk@pitt.edu)            05/24/2005
% Modified: Niels Taatgen (taatgen@cmu.edu) 10/24/2006

%% Change ``a4paper'' in the following line to ``letterpaper'' if you are
%% producing a letter-format document.

\documentclass[10pt,letterpaper]{article}

\usepackage{cogsci}
\usepackage{pslatex}
\usepackage{apacite}


\title{Human Memory and Computer Caches}
 
\author{{\large \bf Avi Press (avipress@berkeley.edu)} \\
  Department of Cognitive Science \\
  Berkeley, CA \\
  \AND {\large \bf Michael Pacer \\
  Department of Psychology \\
  Berkeley, CA

\begin{document}

\maketitle


\begin{abstract}
The abstract should be one paragraph, indented 1/8~inch on both sides,
in 9~point font with single spacing. The heading {\bf Abstract} should
be 10~point, bold, centered, with one line space below it. This
one-paragraph abstract section is required only for standard spoken
papers and standard posters (i.e., those presentations that will be
represented by six page papers in the Proceedings).

\textbf{Keywords:} 
memory, caching, information retrieval
\end{abstract}


\section{General Formatting Instructions}

For standard spoken papers and standard posters, the entire contribution (including figures, references, everything) can be no longer than six pages. For abstract posters, the entire contribution can be no longer than one page. For symposia, the entire contribution can be no longer than two pages.

The text of the paper should be formatted in two columns with an
overall width of 7 inches (17.8 cm) and length of 9.25 inches (23.5
cm), with 0.25 inches between the columns. Leave two line spaces
between the last author listed and the text of the paper. The left
margin should be 0.75 inches and the top margin should be 1 inch.
\textbf{The right and bottom margins will depend on whether you use
  U.S. letter or A4 paper, so you must be sure to measure the width of
  the printed text.} Use 10~point Times Roman with 12~point vertical
spacing, unless otherwise specified.

The title should be in 14~point, bold, and centered. The title should
be formatted with initial caps (the first letter of content words
capitalized and the rest lower case). Each author's name should appear
on a separate line, 11~point bold, and centered, with the author's
email address in parentheses. Under each author's name list the
author's affiliation and postal address in ordinary 10~point type.

Indent the first line of each paragraph by 1/8~inch (except for the
first paragraph of a new section). Do not add extra vertical space
between paragraphs.


\section{Problem Formulation}

Suppose all the data a person or agent has seen and used is the set $D$.
We will have a finite amount of memory $C \subset D$  items ${c_0, c_1, ..., c_k}$. We also have $H$, 
an access history of each item. This can be described as a set of sets that each item was accessed at, ie
${t_0, t_1, ..., t_n}$ for each $c_i \in C$. Whenever an item needs to be accessed,
it must be retrieved from memory. If the item is in $C$, we can retrieve it for a small cost of
$cost_{c}$. If the item is not in cache, we must retrieve it from $D$, for $cost_d$, where
$cost_d > cost_c$.

Our goal then, is to minimize our cost the retrieving items we need from memory.
This means that given $C$ and $H$, we want to minimize our cost
of retrieving sequences of items we need. To do this, we must manage $C$ in order to
maximize retrevals from $C$, and minimize retrevals from $D$, where the cost
to retrieve is much higher.

For computer science purposes, $C$ represents our cache, a small but very fast block of
memory in a computer. $D$ represents disk space, which has far more storage, but is
orders of magnitude slower.

\subsection{The Algorithms}


\subsubsection{Least Recently Used (LRU)}

Perhaps the most simple of caching, LRU evicts the item that was used the least
recently. Because the only information needed to implement LRU is order of uses,
it can easly be implemented with just a list. LRU's strength function can be described by
$$
S_{LRU}(i) = \frac{1} {t_{current} - t_n}
$$
Quite simply, the strength of a memory is proportional to how much time has passed since its last
access. 

We also used a random replacement algorithm to approximate LRU. For this algorithm, 
a random item is selected for eviction.

\subsubsection{Least Frequently Used (LFU)}

Another straightforward approach is to evict the item that has been used least. While often a good approach,
it is less feasible to implement, as LFU requires an ubounded amount of additional memory to store counts
of item uses. LFU's strength function is
$$
S_{LFU}(i) = n
$$
since an item's strength grows the more times it is used. LFU can be very effective when
items are used often but not necessarily in temporal proximity. A major downside
to LFU is that the cache can become littered with items that were once extremely popular, but
might never be used again. 

\subsubsection{LRU-2}

LRU-2 is a specific instance of the LRU-K algorithm. LRU normally doesn't do a good job
of accounting frequency, so LRU-K is a way to approximate LFU without all of the
memory requirements LFU suffers from. LRU-K's strength function is
$$
S_{LRU-2}(i) = \frac{1} {t_{current} - t_{n-k+1}
$$

\subsubsection{2Q}

2Q is an algorithm that tries to find a balance between accounting for recency and frequency
by splitting the cache into two queues. The first queue is managed as an LRU queue. If a hit occurs
in this queue, the item is promoted to the second queue, which is managed as a LFU queue. The LFU 
queue has a predefined maximum size, so items will be evicted from the LFU queue if it is larger
than the predefined size. The strength of an item in the cache will depend on how large the 
LFU queue is. If it is larger than the set limit, the strength of an item will be $S_{LFU}(i)$ if the 
item is in the LFU queue, and infinite otherwise (it wont be evicted). If the LFU queue is
smaller than this threshold, an item's strength is $S_{LRU}(i)$ if it is in the LRU queue, and infinite
otherwise.

One thing that 2Q also does is keep track of the items that were evicted from the LFU queue, by storing
pointers to their address in memory. If one of these evicted items should be used again,
the item can be brought back into memory much faster, making the miss far less costly. Because storing a
pointer takes a very small amount of space, it can reduce the cost of the LFU misses with a small
addition to the memory requirement.

An issue with 2Q is the fact that where LFU queue size has a strict set maximum size, and this maximum size
may be hard or impossible to estimate effectively ahead of time. 

\subsubsection{Adaptive Replacement Cache (ARC)}

ARC is very similar to 2Q but makes generally improving complication. The maximum size of
the LFU queue can adapt to the data incoming. ARC keeps track of items that
have been evicted from each queue. Upon a cache miss of an item that was recently evicted
from one of the queues, ARC will make that queue larger, as it got rid 
of an item that it should have kept. ARC's strength function is identical to
that of 2Q.

\subsubsection{LRFU}

LRFU subsumes both LRU and LFU. Each item has a combined recency-frequency count.
Intuitively, an item strength continually climbs the more it is used, but that
strength decays with time. The exact function and parameter that compute this strength can
vary. As suggested in Lee et. al. (2001), we calculated the strength of an item as:
$$
S_{LRFU}(i) = \sum_i \frac{1}{2}^{\lambda(t_{now} - t_i)}
$$
$\lambda$ is a tunable parameter. For our purposes, .001 worked well.

\subsubsection{The Anderson Model}

Anderson & Schooler (1991) designed a model of human memory to explain phenomena
they witnessed in human data. There were 3 observed phenomena they wished to explain:

* Recency - The more recently an item was used, the more likely it is to be used again.

* Practice - The more times an item has been used, the more likely it is to be used again.

* Spacing - The pattern of prior uses of an item is also a predictor of furutre usage
probability. This pattern can be described by the amount of time passing in between
uses, and how much time passed since this interval occured. 

The reasoning behind spacing may not be intuitive, but it desribes the idea that
acesses of an item may follow a pattern. For the library example, books about taxes are probably
accessed far more frequenly during tax season. Once tax season is over, these books can
probably be brought back to reserves despite their frequency and recency. 

$$
S_{And}(i) = A \sum_i s(t_i)
s(t_i) = t_i^{-d_i}
d_i = max[d_1, b(t_i- t_{i-1})^-d_1]
$$


\subsubsection{Third-Level Headings}

Third-level headings should be 10~point, initial caps, bold, and flush
left. Leave one line space above the heading, but no space after the
heading.


\section{Formalities, Footnotes, and Floats}

Use standard APA citation format. Citations within the text should
include the author's last name and year. If the authors' names are
included in the sentence, place only the year in parentheses, as in
\citeA{NewellSimon1972a}, but otherwise place the entire reference in
parentheses with the authors and year separated by a comma
\cite{NewellSimon1972a}. List multiple references alphabetically and
separate them by semicolons
\cite{ChalnickBillman1988a,NewellSimon1972a}. Use the
et~al. construction only after listing all the authors to a
publication in an earlier reference and for citations with four or
more authors.


\subsection{Footnotes}

Indicate footnotes with a number\footnote{Sample of the first
  footnote.} in the text. Place the footnotes in 9~point type at the
bottom of the page on which they appear. Precede the footnote with a
horizontal rule.\footnote{Sample of the second footnote.}


\subsection{Tables}

Number tables consecutively; place the table number and title (in
10~point) above the table with one line space above the caption and
one line space below it, as in Table~\ref{sample-table}. You may float
tables to the top or bottom of a column, set wide tables across both
columns.

\begin{table}[!ht]
\begin{center} 
\caption{Sample table title.} 
\label{sample-table} 
\vskip 0.12in
\begin{tabular}{ll} 
\hline
Error type    &  Example \\
\hline
Take smaller        &   63 - 44 = 21 \\
Always borrow~~~~   &   96 - 42 = 34 \\
0 - N = N           &   70 - 47 = 37 \\
0 - N = 0           &   70 - 47 = 30 \\
\hline
\end{tabular} 
\end{center} 
\end{table}


\subsection{Figures}

All artwork must be very dark for purposes of reproduction and should
not be hand drawn. Number figures sequentially, placing the figure
number and caption, in 10~point, after the figure with one line space
above the caption and one line space below it, as in
Figure~\ref{sample-figure}. If necessary, leave extra white space at
the bottom of the page to avoid splitting the figure and figure
caption. You may float figures to the top or bottom of a column, or
set wide figures across both columns.

\begin{figure}[ht]
\begin{center}
\fbox{CoGNiTiVe ScIeNcE}
\end{center}
\caption{This is a figure.} 
\label{sample-figure}
\end{figure}


\section{Acknowledgments}

Place acknowledgments (including funding information) in a section at
the end of the paper.


\section{References Instructions}

Follow the APA Publication Manual for citation format, both within the
text and in the reference list, with the following exceptions: (a) do
not cite the page numbers of any book, including chapters in edited
volumes; (b) use the same format for unpublished references as for
published ones. Alphabetize references by the surnames of the authors,
with single author entries preceding multiple author entries. Order
references by the same authors by the year of publication, with the
earliest first.

Use a first level section heading for the reference list. Use a
hanging indent style, with the first line of the reference flush
against the left margin and subsequent lines indented by 1/8~inch.
Below are example references for a conference paper, book chapter,
journal article, technical report, dissertation, book, and edited
volume, respectively.

\nocite{ChalnickBillman1988a}
\nocite{Feigenbaum1963a}
\nocite{Hill1983a}
\nocite{OhlssonLangley1985a}
\nocite{Lewis1978a}
\nocite{NewellSimon1972a}
\nocite{ShragerLangley1990a}


\bibliographystyle{apacite}

\setlength{\bibleftmargin}{.125in}
\setlength{\bibindent}{-\bibleftmargin}

\bibliography{CogSci_Template}


\end{document}
